\documentclass[12pt]{article}

\usepackage{geometry} % see geometry.pdf on how to lay out the page. There's lots.
\geometry{a4paper} % or letter
\geometry{margin=1in}

\title{ PaxosLease: Diskless Paxos for Leases }
\author{ Marton Trencseni }
%\date{}

\begin{document}

\maketitle


\section{ Introduction }
%%%%%%%%%%%%%%%%%%%%%%%%

\subsection{ Failure conditions }

PaxosLease handles the following failure conditions:

\begin{enumerate}
\item Nodes stop and restart
\item Network splits
\item Message loss
\item Message reordering
\item In-transit message delays
\end{enumerate}

PaxosLease does not handle Byzantine failures and messages corruption.

\section{ Preliminaries }
%%%%%%%%%%%%%%%%%%%%%%%%%

\subsection{ Definitions }

\subsubsection{ Actors }
A PaxosLease cell consists of \emph{proposers}, \emph{acceptors} and \emph{learners}. We assume there are $n$ acceptors and any number of proposers and learners. In practice nodes often act as all three, but this is an implementation issue and does not affect the discussion.

\subsubsection{ Messages }
Proposers send \emph{prepare request} and \emph{propose request} messages to acceptors, who reply with \emph{prepare response} and \emph{propose response} messages. Proposers send \emph{learn} messages to learners.

The messages have the following structure:
\begin{enumerate}
\item \emph{$<$prepare request$>$ = $<$ballot number$>$}
\item \emph{$<$prepare response$>$ = $<$ballot number, accepted proposal$>$}
\item \emph{$<$propose request$>$ = $<$proposal$>$}
\item \emph{$<$propose response$>$ = $<$ballot number$>$}
\item \emph{$<$learn$>$ = $<$value$>$}
\end{enumerate}

A \emph{proposal} consists of a \emph{ballot number} and a \emph{value}.

\subsubsection{ State }
Acceptors store the following state information:
\begin{enumerate}
\item \emph{highest ballot promised}: the highest ballot number sent in a prepare response
\item \emph{accepted proposal}: the last proposal accepted
\end{enumerate}

\subsubsection{ Ballot numbers }
Ballot numbers are positive integers generated by proposers. Different proposers generate different ballot numbers. In implementations, this is achieved by choosing ballots numbers such that
$ballot \; number \; mod \; n = proposer \; id$.

\subsubsection{ Chosen value }
A proposal $p$ containing value $v$ is chosen if $ accepted \; proposal = p $ for a majority of acceptors.

\subsection{ Algorithm }

\subsubsection{ Basic flow }

A proposer wishing to propose a value first generates a ballot number and sends prepare request messages to a majority of acceptors.

The acceptor, upon receiving a prepare request with \emph{ballot number}, checks whether $ballot \; number > highest \; ballot \; promised$. If not, the acceptor does not respond to the message.
If yes, the acceptor constructs a prepare response containing the \emph{ballot number} and the currently accepted proposal. If it has not accepted a proposal, it sends special \emph{empty proposal}. The acceptor sets $highest \; ballot \; promised := ballot \; number$ and then sends the prepare response to the proposer.

The proposer, upon receiving prepare responses from a majority of acceptors, examines the responses. If a majority of nodes responded with empty values, the proposer is free to propose its own value. If not, it must propose the value belonging to the highest \emph{accepted ballot number} among the prepare responses. The proposer sends propose requests containing the generated ballot number and the value to the acceptors who responded.

The acceptor, upon receiving a propose request with proposal $p$ containing \emph{ballot number}, checks its \emph{highest ballot promised}. If $ballot \; number < highest \; ballot \; promised$ it ignores the message, else it accepts the message by setting
$accepted \; proposal := p$.
Finally it sends a propose response with the ballot number.

\subsection{ Paxos }

Augmenting the above rules by two requirements we get Paxos. First, proposers, using stable storage, must guarantee that proposals issued at a later time carry higher ballot numbers. Second, acceptors store their state to stable storage before sending responses.

\subsubsection{ Consistency } Paxos guarantees that only one value is chosen.

Suppose two proposals $p$ and $p'$ with ballot numbers $b$ and $b'$ and values $v$ and $v'$ are chosen. Since ballot numbers are unique, if $b = b'$ then $p = p'$ and $v = v'$. Suppose $b < b'$. For value $v'$ to be chosen, $p'$ must be proposed.

\subsubsection{ Lemma } If a proposal $p_0$ with ballot number $b_0$ and value $v_0$ is chosen, then all higher numbered proposals must contain $v_0$.

Proof of lemma. Let $p_1 = (b_1, v_1)$ be the proposal with the lowest ballot number such that $b_0 < b_1$. Let $A_0$ be a majority of acceptors who accepted $p_0$ and let $A_1$ be a majority of acceptors who sent prepare responses to $p_1$. There must be an acceptor $a$ who is in both sets, who must have \emph{first} accepted $p_0$ and \emph{later} replied to the prepare request with $b_1$. In the prepare reply $a$ indicated that it accepted $p_0$, and since $p_1$ is the lowest numbered proposal after $p_0$ the proposer could not have recieved a reply containing a higher numbered proposal, it must have used the value $v_0$ from proposal $p_0$, thus $v_0 = v_1$. Repeating this argument for the next proposal $p_2$, if a majority of nodes sent prepare replys, at least one must have contained $p_0$ or $p_1$, thus $p_2$ must contain the value $v_0$. Continuing \emph{ad infinitum} proves the lemma.

\section{ PaxosLease with perfect clocks}
%%%%%%%%%%%%%%%%%%%%%%%%%%%%%%%%%%%%%%%%%

\subsection{ Acquiring leases }

Augmenting the basic flow with the following rules we get $PaxosLease_{acquire}$ (in the perfectly synchronized clock case):

\begin{enumerate}
\item there is a global maximal lease time $T$ known to all nodes

\item upon restart, nodes do not rejoin the network for $T$ time

\item for proposals issued within time T, proposals issued at a later time carry higher ballot numbers

%\item every proposer is a learner and every learner is a proposer

\item proposers can only propose themselves as lease owners and must honor the global maximal lease time: all values proposed by the kth proposer are of the form $"acquire, k, t_e"$ where $t < t_e < t + T$

\item messages concerning expired leases are discarded: if an acceptor, proposer or learner receives a message with value $"acquire, k, t_e"$ and $t_e < t$ then the message is automatically discarded

\item acceptors use a timer and discard their state: before sending a propose response for a value $"acquire, k, _e"$ the node resets its timer to expire at $t_e$

\item if an acceptor's timer expires at $t_e$, it discards any accepted proposals by setting $accepted \; proposal := empty \; proposal$

\end{enumerate}

Note that nodes do not have to store their state on stable storage, hence the title "diskless Paxos"!

\subsubsection{ Consistency } $PaxosLease_{acquire}$ guarantees that if value $"acquire, k, t_e"$ is chosen at time $t$, then no other value is chosen until time $t_e$.

Suppose two proposals $p$ and $p'$ with ballot numbers $b$ and $b'$ and values $v = "acquire, k, t_e"$ and $v' = "acquire, k', t_e'"$ are chosen at times $t$ and $t'$ such that $t < t' < t_e < t + T$ and $t < t' < t_e' < t' + T$.

Proposal $p$ was issued within the time interval $(t_e - T, t)$, proposal $p'$ was issued within the time interval $(t_e' - T, t')$. By the above inequalities, the union of these intervals is contained in the interval $(t' - T, t')$, which has length $T$. Since $v'$ was proposed within $T$ time of $t$ and proposers guarantee not to reuse ballot numbers for $T$ time, if $b = b'$ then $v = v'$.

Let $A$ be a majority of acceptors who accepted $p$ at time $t$ and let $A'$ be a majority of acceptors who accepted $p'$ at time $t'$.  There must be an acceptor $a$ who is in both sets and since $t < t' < t + T$ node $a$ did not restart in the meantime. Acceptors, within time $T$ accept proposals with increasing ballot numbers, thus $b < b'$.

\subsubsection{ Lemma } If a proposal $p_0$ with ballot number $b_0$ and value $v_0 = "acquire, k, t_e"$ is chosen at time $t$, then all higher numbered proposals chosen at time $t'$, such that $t < t' < t_e$, must contain $v_0$.

Proof of lemma. See proof of previous lemma and note that $a$ could not have restarted.

%Proof of lemma. Let $p_1 = (b_1, v_1)$ be the proposal with the lowest ballot number such that $b_0 < b_1$. Let $A_0$ be a majority of acceptors who accepted $p_0$ and let $A_1$ be a majority of acceptors who sent prepare responses to $p_1$. There must be an acceptor $a$ who is in both sets, who must have \emph{first} accepted $p_0$ and \emph{later} replied to the prepare request with $b_1$. In the prepare reply $a$ indicated that it accepted $p_0$, and since $p_1$ is the lowest numbered proposal after $p_0$ the proposer could not have recieved a reply containing a higher numbered proposal, it must have used the value $v_0$ from proposal $p_0$, thus $v_0 = v_1$. Repeating this argument for the next proposal $p_2$, if a majority of nodes sent prepare replys, at least one must have contained $p_0$ or $p_1$, thus $p_2$ must contain the value $v_0$. Continuing \emph{ad infinitum} proves the lemma.

\subsection{ Extending leases }

$PaxosLease_{acquire, extend}$: The above algorithm can be extended to allow proposers to \emph{extend their leases}. Only the proposer's rule is changed: if a proposer $k$ receives prepare responses from a majority of acceptors and the highest numbered reply contains the value $"acquire, k, t_e"$ such that $t < t_e$ (its lease has not yet expired) then it is free to propose a different value $"acquire, k, t_e'"$ such that $t < t_e < t_e' < t + T$.

\subsubsection{ Consistency } $PaxosLease_{acquire, extend}$ guarantees that if value $"acquire, k, t_e"$ is chosen at time $t$ and value $"acquire, k', t_e'"$ is chosen at time $t'$ such that $t < t' < t_e$ then $k = k'$.

The proof is similar as above. Suppose two proposals $p$ and $p'$ with ballot numbers $b$ and $b'$ and values $v = "acquire, k, t_e"$ and $v' = "acquire, k', t_e'"$ are chosen at times $t$ and $t'$ such that $t < t' < t_e < t + T$ and $t < t' < t_e' < t' + T$.

Proposal $p$ was issued within the time interval $(t_e - T, t)$, proposal $p'$ was issued within the time interval $(t_e' - T, t')$. By the above inequalities, the union of these intervals is contained in the interval $(t' - T, t')$, which has length $T$. Since $v'$ was proposed within $T$ time of $t$ and proposers guarantee not to reuse ballot numbers for $T$ time, if $b = b'$ then $v = v'$.

Let $A$ be a majority of acceptors who accepted $p$ at time $t$ and let $A'$ be a majority of acceptors who accepted $p'$ at time $t'$.  There must be an acceptor $a$ who is in both sets and since $t < t' < t + T$ node $a$ did not restart in the meantime. Acceptors, within time $T$ accept proposals with increasing ballot numbers, thus $b < b'$.

\subsubsection{ Lemma } If a proposal $p_0$ with ballot number $b_0$ and value $v_0 = "acquire, k, t_e"$ is chosen at time $t$, then all higher numbered proposals chosen at time $t'$, such that $t < t' < t_e$, must contain values $"acquire, k', t_e'"$ such that $k = k'$.

Proof of lemma. The proof is similar as the previous two lemmas. Let $p_1 = (b_1, v_1)$ be the chosen proposal (at time $t'$ such that $t < t' < t_e$) with the lowest ballot number such that $b_0 < b_1$. Let $A_0$ be a majority of acceptors who accepted $p_0$ and let $A_1$ be a majority of acceptors who sent prepare responses to $p_1$. There must be an acceptor $a$ who is in both sets, who must have \emph{first} accepted $p_0$ and \emph{later} replied to the prepare request with $b_1$. In the prepare reply $a$ indicated that it accepted $p_0$, and since $p_1$ is the lowest numbered proposal after $p_0$ the proposer could not have recieved a reply containing a higher numbered proposal, it must have used the value $v_0$ from proposal $p_0$, or if the proposer was node $k$, issued another proposal for itself. Repeating this argument for the next proposal $p_2$, if a majority of nodes sent prepare replys, at least one must have contained $p_0$ or $p_1$, thus $p_2$ must contain a value proposing $k$ as the leaseholder. Continuing \emph{ad infinitum} proves the lemma.

\subsection{ Releasing leases }

In the formulation of PaxosLease (and Paxos) so far, a value was chosen if a majority of acceptors accepted a proposal. In practice, nodes must be able to \emph{learn} that a value was chosen. In classical Paxos, a proposer learns that a value was chosen when it receives \emph{propose responses} for a proposal $p$ from a majority of acceptors. Then it can choose to inform learners by sending them $learn$ messages.

In the full formulation of PaxosLease, $PaxosLease_{acquire, extend, release}$, we change the definition of chosen value: a proposal $p$ containing value $v$ is chosen if the proposer receives propose response messages from a majority of acceptors.
Note that this definition is stronger then the previous, so all proofs presented so far remain valid.

Now we can complete PaxosLease by introducing special $"release, k, t_r"$ values. Both the proposer's and acceptor's rules are amended.

\begin{enumerate}

\item An acceptor, receiving a proposal $p$ with ballot number $b$ (such that $highest \; ballot \; promised <= b$) and value $"release, k, t_r"$, if currently holding accepted value $"acquire, k, t_e"$ where $t_r < t_e < t$ must change $t_e$ to $t_r$ in its accepted value and reset its timer to $t_r$.

\item Proposer $k$, at time $t$ learning that its value $"acquire, k, t_e"$ was chosen, where $t < t_e < t + T$, may either extend its lease as described above or release its lease at time $t'$ by issuing a proposal with value $"release, k, t_r"$ where $t < t' <  t_r < t_e < t + T$.

\end{enumerate}

\subsubsection{ Consistency } $PaxosLease_{acquire, extend, release}$ guarantees that if value $"acquire, k, t_e"$ is chosen at time $t$ and value $"acquire, k', t_e'"$ is chosen at time $t'$ such that $t < t' < t_e$ and proposer $k$ did not issue proposals containing $release$ messages in the meantime, then $k = k'$.

The proof is almost the same as above. Suppose two proposals $p$ and $p'$ with ballot numbers $b$ and $b'$ and values $v = "acquire, k, t_e"$ and $v' = "acquire, k', t_e'"$ are chosen at times $t$ and $t'$ such that $t < t' < t_e < t + T$ and $t < t' < t_e' < t' + T$.

Proposal $p$ was issued within the time interval $(t_e - T, t)$, proposal $p'$ was issued within the time interval $(t_e' - T, t')$. By the above inequalities, the union of these intervals is contained in the interval $(t' - T, t')$, which has length $T$. Since $v'$ was proposed within $T$ time of $t$ and proposers guarantee not to reuse ballot numbers for $T$ time, if $b = b'$ then $v = v'$.

Let $A$ be a majority of acceptors who accepted $p$ at time $t$ and let $A'$ be a majority of acceptors who accepted $p'$ at time $t'$.  There must be an acceptor $a$ who is in both sets and since $t < t' < t + T$ node $a$ did not restart in the meantime. Acceptors, within time $T$ accept proposals with increasing ballot numbers, thus $b < b'$.

\subsubsection{ Lemma } If a proposal $p_0$ with ballot number $b_0$ and value $v_0 = "acquire, k, t_e"$ is chosen at time $t$, then all higher numbered proposals chosen at time $t'$, such that $t < t' < t_e$, must contain values $"acquire, k', t_e'"$ such that $k = k'$, \emph{provided proposer $k$ did not issue $release$ proposals between $t$ and $t'$}.

The proof of the lemma is the same as in the $PaxosLease_{acquire, extend}$ case.

\subsection{ Learning }

What remains is to inform learners of leases. Only the proposer's algorithm changes:

\begin{enumerate}

\item If proposer $k$ receives propose response messages from a majority of acceptors containing value $v = "acquire, k, t_e"$, then it sends this learn messages containing $v$ to the learners. Learners, upon receiving this value, assume that $k$ holds the lease until time $t_e$.

\item If proposer $k$, who is currently holding the lease wants to release it at time $t_r$, it sends learn messages containing the value $v = "release, k, t_r"$ to learners \emph{before issuing the proposal for $v$}. Learners, upon receiving this value, if they knew $k$ currently holds the value, assume that $k$ holds the lease until time $t_r$.

\end{enumerate}

If the proposer sends the learn messages containing $"release..."$ values to the learners and then fails, then there may be an interim period (of length less than $T$ time) when, according to the learners, noone holds the lease, but no other proposer will be able to acquire it. This does not affect correctness.

On the other hand, if the proposer does not fail and sends enough propose messages containing $"release..."$ values, other proposers will be able to acquire the lease before the proposer's previously held lease's expire time.

\section{ PaxosLease with skewed clocks }
%%%%%%%%%%%%%%%%%%%%%%%%%%%%%%%%%%%%%%%%%

So far, we have assumed that all nodes know some global time $t$. In practice this is not true, so we assume the $kth$ node has a local clock $c_k(t)$ with a maximal skew $S$ such that

\begin{enumerate}

\item $|c_k(t) - c_j(t)| < S$ for all times $t$ and nodes $k, j$

\item the global maximal lease time $T$ is chosen such that $S < T$.

\end{enumerate}

Changes are required to handle clock skew:

\begin{enumerate}

\item When restarting, proposers and acceptors must wait for $T + S$ time

\item If a proposer $k$ wants to acquire a lease until local time $t_e$. It must take into account that acceptors' clocks may be ahead of its own by $S$ time. It does this by proposing the value $"acquire, k, t_e'"$, where $c_k(t) + S < t_e' = t_e + S < c_k(t) + T$. This ensures that an acceptor, even if its clock is ahead of k's by $S$ time, does not delete the accepted value before k's local time reaches $t_e$.
 
\item After a proposer has acquired a lease until time $t_e$, it may notify learners, but it must take into account that learners clocks' may be behind the acceptor's clock by $S$ time. If value $"acquire, k, t_e'"$ ($t_e' = t_e + S$) is accepted by a majority of acceptors, proposer $k$ sends learn messages containing time $t_e$. This ensures that a learner, even if its clock is behind the acceptors' by $S$ time, does not falsely believe that $k$ owns the lease after acceptor's local time reaches $t_e'$ and they discard their state.

\item If proposer $k$ wishes to release a lease at local time $t_r$ that it holds until local time $t_e$ (the previously accepted value is $"acquire, k, t_e + S"$), it must issue proposal $"release, k, t_r'"$ where $c_k(t) + S < t_r' = t_r + S < t_e$. This ensures that an acceptor, even if its clock is ahead of k's by $S$ time, does not delete the accepted value before k's local time reaches $t_e$.

\item Proposals must send $"release..."$ values to learners before proposing the value. If it wishes to release the resource at local time $t_r$, it sends the value $"release, k, t_r"$ to the proposers. Since it will propose the value $"release, k, t_r' = t_r + S"$ to the acceptors, this ensures that learner's do not falsely believe $k$ holds the lease after acceptor's local time reashes $t_r'$ and they discard their state.

\end{enumerate}

\section{ Leases for many resources }
%%%%%%%%%%%%%%%%%%%%%%%%%%%%%%%%%%%%%

The algorithm defines lease actions concerning a single resource $R$. In practice, nodes wish to deal with several resources, such as write locks for files in a distributed filesystem. The solution is to run several independent instances of PaxosLease in parallel, each with its own proposer, acceptor and learner states. To identify leases in messages, prefix each messages with a \emph{resources identifier}.

A node acting as proposer, acceptor and learner (a common setup in distributed systems) requires ~ 50-100 bytes of memory for each PaxosLease instance, or 10-20 million resource leases per gigabyte of main memory.

\end{document}